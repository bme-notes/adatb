\section{Tranzakció kezelés}

\textbf{ACID} %TODO

atomicity - tranzakció lefutása úgy hogy minden lefut belőle

konzisztencia - az adatbázisba csak a sikeresen lefutott tranzakciók vannak

Def: Soros ütemezés $\Rightarrow$ izolált tranzakció

DE izolált $\nRightarrow$ Soros \\[2pt]


\textbf{Átlapolódásból} adodó problémák:
 \begin{enumerate} \setlength\itemsep{-0.5mm}
 	\item Dirty read
 	\item Phantom read
 	\item Lost update
 	\item Non-repetable read
 \end{enumerate}
\textcolor{red}{ A \textbf{sorosítással} ezek a problémák elkerülhetők.}\\[2pt]


Problémák a zárakkal - Deadlock megoldási lehetőségek \\[2pt]

%TODO MI az a tuple?, b* fa

\begin{definicio}{Lavina}
	Több piszkos adat olvasása, mely lavina szerűen terjed\\[-3pt]
\end{definicio}

\begin{definicio}{Agresszív protokoll}
Optimista konkurenciakezelés. Egy protokoll agresszív, ha megpróbál olyan gyorsan lefutni, amennyire csak lehetséges, nem törődve azzal, hogy ez esetleg aborthoz is vezethet. ( Keveset foglalkozik a zárakkal)

PL: Zárakat szigorúan olvasás/írás előtt kéri.\\[-3pt]
\end{definicio}

\begin{definicio}{Konzervatív protokoll}
Pesszimista konkurencia kezelés. Egy protokoll konzervatív, ha megkísérli elekrülni az olyan tranzakciók futását amelyek nem biztos, hogy eredményesek lesznek (Sokat foglalkozik a zárakkal)

PL:\\[-21pt]

\begin{enumerate} \setlength\itemsep{0mm}
	\item Szigorú 2PL (Lavina mentes és sorosítható)
	\item Az összes zárat előre kéri, és csak akkor indul, ha már az összeset megkapta (Patmentes )
	\item csak akkor kap meg egy zárat, ha az előtte senkinek nem kell az adatelemhez tartozó várakozó sorban. ( Éhezésmentes )
\end{enumerate}

\end{definicio}

\textbf{Ellenőrzési pontok (checkpoint) }\setlength\itemsep{0.5mm}

\begin{enumerate}
	\item Ideiglenesen megtiltjuk új tranzakció indítását, és megvárjuk, amíg minden tranzakció befejeződik vagy abortál.
	\item Megkeressük azokat a memóriablokkokat, amelyek módosultak, de még nem kerültek a háttértárba kiírásra,
	\item Ezeket a blokkokat visszaírjuk a háttértárra.
	\item Ellenőrzési pont (checkpoint) naplóba
	\item naplót kiírjuk a háttértárra
\end{enumerate}

\textbf{Médiahibák elleni védekezés}

\begin{itemize}
	\item Rendszeres mentések (backup).
	\item Az adatbázis fájlok duplikálása lehetőleg egy másik fizikai eszközön, néha más földrajzi helyen (\textit{elosztott adatbázisok})
\end{itemize}

\textbf{Fa protokoll} - pl B* fa

\textbf{Figyelmeztető protokoll} - pl: Egy egész tábla lockolása relációs adatbázisba

\subsection{Időbélyeges tranzakciókezelés}

	Sorosíthatóság biztosításának egy másik módja.Akkor praktikus, ha a tranzakciók között kevés a potenciális sorosítási konfliktus.

	Mindig \textbf{Deadlock mentesek}

	\begin{center}
		\begin{tabular}{|l||c|c|}
 			 \hline
 			 & T olvasni akar & T írni akar\\ \hline \hline
 			 t(T) $>$ R(A) & \checkmark & \checkmark \\
 			 t(T) $>$ W(A) & R(A) := t(T) & W(A) := t(T) \\ \hline

 			 t(T) $>$ R(A) & $\times$ & $\times$ abort T \\
 			 t(T) $<$ W(A) & & \textcolor{green}{Thomas} szabály: T mehet \\ \hline

 			 t(T) $<$ R(A) & \checkmark & $\times$ \\
 			 t(T) $>$ W(A) & & \\ \hline

 			 t(T) $<$ R(A) & $\times$ & $\times$ \\
 			 t(T) $<$ W(A) & & \\ \hline
 		\end{tabular}

 		\small Thomas szabály: T tovább mehet, hiszen senki nem használja amit ő írt bele, mert már előbb felülírták \normalsize
 	\end{center}

Összefoglalva:

Olvasás: $t(T) \geq W(A)$ \checkmark

Írás: $t(T) \geq R(A) \land t(T) \geq W(A)$ \checkmark

Thomassal írás: $t(T) \geq R(A) $


\subsubsection{Tranzakcióhibák és az időbélyegek}

	Előfordulhat \textbf{inkonzisztencia}(lavinákból adódóan)\\[2pt]
	Megoldás:

		\begin{itemize}
			\item Elfogadjuk a lavinákat, hiszen az időbélyeges tranzakciókezelést tipikusan olyan környezetben használjuk, ahol kevés az abort, tehát a lavina még kevesebb
			\item Megakadályozzuk a piszkos adat olvasását pl: Commit ponting nem írunk adatbázisba

				Lépései az alábbiak lennének:

				\begin{enumerate}
					\item Módosítások csak a munkaterületen végezve
					\item Tranzakció eléri a készpontját
					\item írások véglegesítése az adatbázison
				\end{enumerate}
		\end{itemize}
Illetve előfordulhat, hogy a "lassú tranzakciók"ból adódóan, 1 adategységet hibásan olvasnak ki.

	Megoldás: \textbf{Zár}ak használata

\subsubsection{Verziókezeléssel}

	Minden adatelem írásakor az előzőt megőrizzük.

	\textbf{Abort} kell, ha T írni akar és $W(A_i) < t(T) < R(A_i)$ \quad Tehát ha olyan adatot olvastunk, melyet most kéne a T tranzakciónak beírnia.

\subsubsection{Elosztott eset}

 \begin{enumerate}
 	\item Időbélyeg kiegészítése egyedi azonosítóval csomóponttól függően
 	\item Minden lokális példányon el kell végezni a módosítást
 	\item PL: Wall protokoll szerint döntendő el hogy $READ\ A$ érvényes-e (Egy $R(A_i)$ és $W(A_i)$ vizsgálat)
 \end{enumerate}
