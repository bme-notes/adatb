
\section{Fizikai Adatszervezés}

	\begin{enumerate}
		\item Heap szervezés
			\begin{itemize}
				\item Lineáris keresés : $\dfrac{b_r + 1}{2}$
			\end{itemize}

		\item \textbf{Vödrös Hash}
			\begin{itemize}
				\item NINCSENEK KULCSOK TÁROLVA
				\item Vödrökön belül lineáris keresés
			\end{itemize}

		\item \textbf{Indexelt állományok}

			\begin{itemize}
				\item	Támogatja a \textbf{több kulcs} szerinti keresést - Ez esetben több index állomány

				\item Index állományt mindig rendezve tartjuk

				\begin{enumerate}
					\item \textbf{Sűrű index}

							Mutató minden egyes adatrekordra.

							Keresés $log_2(b_r)$ - $b_r$ = indexállományblokkok száma

							\forceindent •\textbf{B* fa} $\Longrightarrow$ $log_k(b_r)$ el arányos keresési idő, ahol k a szintek száma


					\item \textbf{Ritka index}

							\begin{itemize}
								\item - Plusz helyigény
								\item - Egyel több lapelérés
								\item - Több adminisztrációval jár a karbantartása
								\item + Az adatállományt nem kell rendezetten tartani
								\item + Meggyorsítja a rekordelérést, mert a ritka index mérete jóval kissebb is lehet, mint egy sűrű index
								\item Támogatja a több kulcs szerinti keresést
								\item Az adatállomány rekordjai szabaddá tehetők ,ha minden további hivatkozás a sűrű indexen keresztül történik
							\end{itemize}
				\end{enumerate}
			\end{itemize}
	\end{enumerate}
